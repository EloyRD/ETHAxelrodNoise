\section{Implementation}

As described in section~\ref{sec:intro}, the tournament is a repeated prisoners dilemma. The payoff matrix for this kind of game is shown in table~\ref{tab:rewardmatrix}.\\

To make the simulation more realistic, there is noise added to it. Noise means that defection can be transmitted as cooperation and vice versa. The noise applied on cooperation and defection was varied independently. The two noise levels are set independently to $0\%$, $5\%$, $10\%$ and $15\%$. The noise only changed the information the players received, but not their payoff. For each combination of noise, we performed a tournament with $20000$ rounds. In each round, every player plays agains all others and himself in a round robin. To make the decisions, the players are provided with all the decisions made in the previous rounds by all players. The players do not have information about the noise level or the duration of the tournament. \\

\begin{table}[h]

 \begin{center}
\caption{Reward Matrix}\label{tab:rewardmatrix} \vspace{3mm}
\begin{tabular}{|l|c|c|}

\hline
   & Player B cooperates & Player B defects \\
  \hline
  Player A cooperates & A:3 B:3 & A:0 B:5 \\
 \hline
  Player A defects & A:5 B:0 &A:1 B:1 \\
 \hline
\end{tabular}
 \end{center}

\end{table}

The basic structure of a player is shown in listing~\ref{playertemplate}. 

\begin{itemize}\item[]\lstinputlisting[caption=Template for each player,label=playertemplate]{../matlab/playertemplate.m}\end{itemize}

An example of such a player, \verb0TFT0 in this case, is given in listing~\ref{tftmatlab}. 

\begin{itemize}\item[]\lstinputlisting[caption=Example of a player,label=tftmatlab]{../matlab/player4.m}\end{itemize}

The noise is defined as followed:

\begin{definition}[Noise]
\begin{itemize}
	\item[] \hspace{3mm}
	\item Noise$1$ = probability for cooperations gets received as defections
	\item Noise$2$ = probability for defections gets received as cooperations
\end{itemize}
\end{definition}

By means of this, we can chose an "optimistic" noise and a "pessimistic" noise. The optimistic ones is much more pardoning and some of the defections taken are not transmitted to the opponent. The other typ of noise is really pessimistic and misunderstands some of the cooperations as defections. The players just have access to the corrupted and noisy decisions, for evaluation the real decisions are also kept.\\

For evaluation and visualization of the results of the tournament we have written the script \verb0show_data.m0. With this script, there are many possibilities to plot all parts of the results. In linsting~\ref{showdata} the description in the header of the script is shown.

\begin{itemize}\item[]\lstinputlisting[caption=Possibilities of visualization,label=showdata]{../matlab/show_data_header.m}\end{itemize}

 
